\documentclass[10pt]{article}
% \usepackage[UTF8]{ctex}

\usepackage[utf8]{inputenc} % allow utf-8 input
\usepackage{amsthm}
\usepackage{amsmath,amscd}
\usepackage{amssymb,array}
\usepackage{amsfonts,latexsym}
\usepackage{graphicx,subfig,wrapfig}
\usepackage{times}
\usepackage{psfrag,epsfig}
\usepackage{verbatim}
\usepackage{tabularx}
\usepackage[pagebackref=true,breaklinks=true,letterpaper=true,colorlinks,bookmarks=false]{hyperref}
\usepackage{cite}
\usepackage{algorithm}
\usepackage{multirow}
\usepackage{caption}
\usepackage{algorithmic}
%\usepackage[amsmath,thmmarks]{ntheorem}
\usepackage{listings}
\usepackage{color}
\usepackage{bm}
\usepackage{enumitem}
\usepackage{longtable}

\allowdisplaybreaks[4]
\linespread{1.25}

\newtheorem{thm}{Theorem}
\newtheorem{mydef}{Definition}

\DeclareMathOperator*{\rank}{rank}
\DeclareMathOperator*{\trace}{trace}
\DeclareMathOperator*{\acos}{acos}
\DeclareMathOperator*{\argmax}{argmax}


\renewcommand{\algorithmicrequire}{ \textbf{Input:}}
\renewcommand{\algorithmicensure}{ \textbf{Output:}}
\renewcommand{\mathbf}{\boldsymbol}
\newcommand{\mb}{\mathbf}
\newcommand{\matlab}[1]{\texttt{#1}}
\newcommand{\setname}[1]{\textsl{#1}}
\newcommand{\Ce}{\mathbb{C}}
\newcommand{\Ee}{\mathbb{E}}
\newcommand{\Ne}{\mathbb{N}}
\newcommand{\Se}{\mathbb{S}}
\newcommand{\norm}[2]{\left\| #1 \right\|_{#2}}

\newenvironment{mfunction}[1]{
	\noindent
	\tabularx{\linewidth}{>{\ttfamily}rX}
	\hline
	\multicolumn{2}{l}{\textbf{Function \matlab{#1}}}\\
	\hline
}{\\\endtabularx}

\newcommand{\parameters}{\multicolumn{2}{l}{\textbf{Parameters}}\\}

\newcommand{\fdescription}[1]{\multicolumn{2}{p{0.96\linewidth}}{

		\textbf{Description}

		#1}\\\hline}

\newcommand{\retvalues}{\multicolumn{2}{l}{\textbf{Returned values}}\\}
\def\0{\boldsymbol{0}}
\def\b{\boldsymbol{b}}
\def\bmu{\boldsymbol{\mu}}
\def\e{\boldsymbol{e}}
\def\u{\boldsymbol{u}}
\def\x{\boldsymbol{x}}
\def\v{\boldsymbol{v}}
\def\w{\boldsymbol{w}}
\def\N{\boldsymbol{N}}
\def\X{\boldsymbol{X}}
\def\Y{\boldsymbol{Y}}
\def\A{\boldsymbol{A}}
\def\B{\boldsymbol{B}}
\def\y{\boldsymbol{y}}
\def\cX{\mathcal{X}}
\def\transpose{\top} % Vector and Matrix Transpose

%\long\def\answer#1{{\bf ANSWER:} #1}
\long\def\answer#1{}
\newcommand{\myhat}{\widehat}
\long\def\comment#1{}
\newcommand{\eg}{{e.g.,~}}
\newcommand{\ea}{{et al.~}}
\newcommand{\ie}{{i.e.,~}}

\newcommand{\db}{{\boldsymbol{d}}}
\renewcommand{\Re}{{\mathbb{R}}}
\newcommand{\Pe}{{\mathbb{P}}}

\hyphenation{MATLAB}

\usepackage[margin=1in]{geometry}

\begin{document}

\title{Numerical Optimization, 2021 Fall\\Homework 6}
\author{Your Name\\Your ID}
\date{Due 23:59 (CST), Nov. 27, 2021 \\(NOTE: Homework will not be accepted after this due for any reason.)\\}
\maketitle

%===============================

\section{Convex Basics}
Prove that $f: \mathbb{R}^n \rightarrow \mathbb{R}$ is affine if and only if $f$ is both convex and concave.

%===============================

\section{Newton's Method}
Consider minimizing $\frac{1}{2} x^T Ax+b^Tx+c$, $A$ positive definite.  Prove that Newton's method will converge to the optimal in one step starting at an arbitrary initial point $x_0$.

\section{Coding}
In this example we want to compute the minimum of the non-convex bivariate Rosenbrock function "banana function"
$$
f(x,y)=(1-x)^2+100(y-x^2)^2.
$$
It is notorious in optimization examples because of the slow convergence most methods exhibit when trying to solve this problem. And the domain of the above Rosenbrock function is $x \in [-2,2]$, $y \in [-2,2]$.
\begin{enumerate}
    \item Show the gradient of $f(x,y)$.
    \item Plot the 3D-surface of $f(x,y)$ on its domain.
    \item Use Gradient Descent method with \textit{Armijo Backtracking Line Search} to minimize $f(x,y)$.
    \item Use Newton's method with \textit{Armijo Backtracking Line Search} to minimize $f(x,y)$.
\end{enumerate}

Please start at the point $(x,y) = ( -1.2, 1 )$ to initiate your algorithms. For each case, draw the curves of the step size $\alpha^k$ and $\| \nabla f(x^k,y^k) \|_{\infty}$ of the two algorithms, with respect to the iteration number $k$. 

Your program may be implemented with Python or MATLAB. Please paste your curves down below, then put your code in an independent file (We encourage you to submit a *.mlx (by MATLAB Live Editor) or *.ipynb (by Jupyter) file to answer the question. Others are fine, of course) and submit it to Blackboard alongside the PDF. 

\textbf{Hint:} You may set initial step size $\alpha_0 = 1$, constant $c_1 = 10^{-4}$, and termination condition as $\| \nabla f(x^k,y^k) \|_{\infty} \leq 10^{-4}$.
\end{document}